\title{Technischer Bericht \\ Tech4Germany}

\documentclass[10pt]{article}
\usepackage[ngerman]{babel}
\usepackage[margin=1in]{geometry}

\begin{document}
\maketitle

\section{Einleitung}

Dieses Dokument gibt einen Kurz-Überblick über die technischen Bedingungen
dieses Projekts. Dabei werden die technologischen Überlegungen angerissen. Bei
Rückfragen wenden Sie sich bitte an Manuel Lang und/oder Florian Zechmeister.
Insbesondere wird in diesem Dokument auf den vorgesehenen Datenbestand sowie
auf die Orientierung, die technisch wesentlich komplexer als die vorhergesene
gefilterte Suche ist und auch im implementierten Prototyp exemplarisch gezeigt
wird, eingegangen.

\section{Datenbestand}

Weiterbildungsangebote sollen in die Anwendung integriert werden können. Dabei
ist vorgesehen, dass diese von verschiedenen Beteiligten sowie über verschiedene
Wege hinzugefügt werden können. Bei den Beteiligten kann es sich um all
diejenigen handeln, die Interesse haben Angebote einzustellen, bspw.
Kursanbieter, Bildungsträger oder Plattformanbieter. Dabei ist im aktuellen
Stand eine Schnittstelle vorgesehen, sodass existierende Angebote ohne
zusätzliche Komplexität integriert werden. Zusätzlich ist im Frontend ein
Formular vorgesehen, sodass einzelne Weiterbildungsangebote auch ohne die
Schnittstelle direkt anzusprechen direkt integriert werden können. Eine
Authentifizierung ist bisher noch nicht implementiert, wird aber auf jeden Fall
benötigt, um eine Zertifizierung der Angebote sicherzustellen. Neben den
Weiterbildungsangeboten werden zur Orientierung auch Berufs-Beschreibungen
benötigt, die entweder aus aktuellen Berufs-Definitionen (z.B. KURSNET), aus
User-Generated Content oder aus Stellenausschreibungen extrahiert werden können.

\section{Orientierung}

Die gezeige Anwendung beinhaltet verschiedene Aspekte zur Orientierung von
Nutzer*innen. Insbesondere ist diese Funktionalität für unsere ermittelten
Personas der ambitionierten Aufsteigerin und des unsicheren Umsteigers
hilfreich. Da allerdings diese Personas verschiedene Ziele verfolgen, haben wir
dies auch in der Anwendung berücksichtigt. So kann sich die ambitionierte
Aufsteigerin nach Angabe ihres aktuellen Tätigkeitsfelds innerhalb eines Raumes
Kurse, die nahe an ihrer derzeitigen Position liegen, explorieren. Der unsichere
Umsteiger dagegen kann ohne Vorgabe der aktuellen Tätigkeit, sondern
ausschließlich basierend auf einer Vorauswahl an Branchen, verschiedene Berufe
explorieren. Diese Auftrennung ist sinnvoll, da sich die Aufsteigerin innerhalb
eines gegebenen Feldes weiterbilden möchte, wobei der Umsteiger verschiedene
Bereiche erkunden möchte. Diese Erkenntnisse stammen aus unseren Interviews mit
Nutzer*innen. Im Folgenden wird die räumliche Exploration des Umsteigers
eingegangen, wobei zunächst ein Ähnlichkeitsmaß für Berufe definiert werden
muss.

\subsection{Ähnlichkeit von Berufen}

Die Bundesagentur für Arbeit stellt in ihrem Portal BERUFENET Definitionen von
Berufen zur Verfügung. Um eine Ähnlichkeit zwischen diesen zu ermitteln, muss
die textuelle Form in einen Merkmals-Vektor überführt werden, der genauere
Informationen zu den einzelnen Berufen liefert. Zwar könnte man die
Levenshtein-Distanz verwenden, allerdings verwendet diese keinerlei
Informationen über die eigentlichen Berufsinhalte.\\

 Anhand der Berufs-Bezeichnungen können durch vortrainierte Modelle wie bspw.
 Word2Vec oder GloVe bereits Merkmals-Vektoren erzeugt werden. Diese Modelle
 erstellen Merkmals-Vektoren anhand von Attributen, die anhand von verschiedenen
 Korpussen wie bspw. den Wikipedia-Daten gelernt wurden. Auch wir haben diese
 Verfahren verwendet um Merkmale für Berufe zu generieren, allerdings verfügen
 diese über keine kontextuellen Informationen zu einem Beruf, d.h. die
 Tätigkeiten sowie verschiedene Berufsbeschreibungen bleiben unberücksichtigt.
 Ein anderer Nachteil dieser Verfahrer ist, dass ein Merkmals-Vektor stets nur
 für ein einzelnes Wort generiert werden kann. Auch wenn die Vektoren über
 verschiedene Wörter gemittelt werden können, gehen sehr viele Informationen
 verloren.\\

Um zusätzliche Informationen zu generieren, haben wir deshalb die
Tätigkeitsbeschreibungen analysiert. Dabei lässt sich mit einem Bag-of-Words
Ansatz ein Korpus generieren, der alle relevanten Wörter durch Verwendung eines
Stemmers in ihrer Rohform beinhaltet. Zu Beachten ist dabei, dass häufige Wörter
rausgefiltert werden müssen, um die Relevanz der einzelnen Merkmale nicht zu
gefährden. So müssen Bindewörter und Pronomen extrahiert werden, wofür sich
bspw. die Python-Bibliothek NLTK sehr gut eignet. Nach der Erstellung des
Korpuses können nun die Merkmals-Vektoren der einzelnen Berufe berechnet werden.
Dazu kann ein TfidfVectorizer bspw. von Scikit Learn verwendet werden, der die
relative Häufigkeit der gestemmten Wörter innerhalb einer Tätigkeitsbeschreibung
betrachtet.\\

So können Merkmals-Vektoren für Berufts-Tätigkeiten bestimmt werden, die sehr
hochdimensional sind. Da auch die meisten Tätigkeitsbeschreibungen nur einen
Bruchteil des Korpuses abdecken, eignet sich die euklidische Distanz ($d_e(p,q)
= \sqrt{\sum_{i=1}^n (p-q)^2}$) nicht, um Ähnlichkeiten (bzw. Distanzen)
zwischen den Berufen zu berechnen. Stattdessen eignet sich die Kosinus-Distanz
($d_c(p,q) = 1 - cos(\theta) = \frac{p \cdot q}{||p|| ||q||}$), da dieser statt
dem Pfad zwischen $p$ und $q$ den räumlichen Winkel zwischen diesen betrachtet.
Um diese Merkmals-Repräsentation greifbar zu machen, eignen sich verschiedene
Algorithmen, die die Dimensionalität des Merkmals-Raums reduzieren. In unserer
beispielhaften Implementierung haben wir T-distributed Stochastic Neighbor
Embedding verwendet, da dieses Verfahren eine ansprechende Visualisierung als
bspw. eine PCA oder eine LDA liefern.\\

Damit die Distanzen nicht zur Laufzeit bestimmt werden müssen, exportieren wir
eine Distanzmatrix, die die paarweise Distanz zweier Embeddings speichert. So
kann diese Matrix beim Starten der Anwendung geladen werden und so direkt auf
`dist\_matrix[i][:]' zugegriffen werden, um die paarweisen Distanzen des
Embeddings $i$ zu den anderen Embeddings zu bestimmen. Mit diesen Distanzen kann
nun innerhalb des hochdimensionalen Raumes navigiert werden.

\subsection{Explorative Navigation durch den Berufsraum}

Der Berufsraum ist sehr hochdimensional, d.h. eine Navigation durch diesen ist
sehr komplex. Beginnend mit ausgewählten Branchen können hinterlegte Berufe, die
exemplarisch die gegebenen Branchen repräsentieren, geladen werden. Über diese
wird dann der Durchschnitts-Vektor berechnet, um einen Start-Punkt im Raum zu
generieren. Für diesen Start-Punkt können nun die Nachbarn berechnet werden,
indem die Kosinus-Distanz zwischen dem gemittelten Vektor und den Embeddings der
einzelnen Berufe berechnet wird. Um eine performente Anwendung zu generieren,
ist die Berechnung der Distanz nicht zwischen allen Paaren zu empfehlen, weshalb
durch die Auswahl jedes bspw. fünften Berufes eine zufällige Komponente in die
Auswahl der Optionsvorschläge integriert wird und gleichzeitig die Ladezeit der
Anwendung stark reduziert wird. Die resultierenden Berufsvorschläge innerhalb
der Optionen sollten erneut stark zufällig gewählt werden, um eine flexible
Exploration zu ermöglichen. Die vorgeschlagenen Top 5 Berufe dagegenen sollten
als nächste Nachbarn des aktuellen Punktes im Berufsraum generiert werden, um
auch wirklich die naheliegendsten Berufe abzudecken.


\end{document}
